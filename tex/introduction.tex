An event driven approach is used to simulate a gas of hard discs in two dimensions. The framework built is used to study statistical properties of the gas at and towards equilibrium. The numerical results are compared with well known results from classical statistical mechanics. 

The central algorithm for event-driven simulations is the following \cite{event_sim}:

\begin{algorithm}[H]
	Set velocities and positions of all particles in the gas\;
	Choose a stop criterion\;
	\For{each particle in gas}{
		Calculate if and when the particle will collide with all the other particles and the walls\;
		Store all the collision times\;
	}
	\While{not reached stop criterion}{
		Identify the earliest collision\;
		
		\eIf{collision is valid}{
			Move all particles in straight lines until the earliset collision\;
			\For{each particle involved in collision}{
				Calculate if and when the particle will collide with all the other particles and the walls\;
				Store all the collision times\;
			} 
		}{
			Discard collsion\;
		}
	}
	\caption{Event driven simulation of a gas.}
\end{algorithm}

In the above algorithm, a collision is \textit{valid} if the particle(s) involved in the collision has \textit{not} collided since the time the collision time was stored. 
