We repeat the exact same procedure as in section \ref{sec:eq}, except that we make half of the masses $4$ times as big as the rest. The velocity distributions after equilibrium is reached is plotted in \ref{fig:dist_3}.

\begin{figure}[htb]
	\centering
	\includegraphics[width=\textwidth]{../fig/distribution_2}
	\caption{Distribution of velocities in a gas of $50000$ particles. The red distributions correspond to the heavy particles, while the blue correspond to the lighter particles.}
	\label{fig:dist_3}
\end{figure}

The average speed and kinetic energy is shown in table \ref{tab:averages}. These results suggests that although the masses are different, the mixture of the gases eventually reach equilibrium.  

\begin{table}[htb]
	\centering 
	\caption{Average speed and kinetic energy for the light and heavy particles}
	\begin{tabular}{ccc}
		\hline
		\textbf{Mass }& \textbf{Average speed} & \textbf{Average kinetic energy} \\
		\hline 
		$m$   & 0.1400 & 0.1259 \\
		$4m$  & 0.0699 & 0.1241 \\
		\hline 
	\end{tabular}
	\label{tab:averages}
\end{table}
