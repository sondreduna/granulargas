\subsubsection{Distributing the discs randomly in $[0,1]\times[0,0.5]$}

To distribute the particles randomly in the box defined by 
$$
\{(x,y)\in \RR^2 : 0 < x < 1, 0 < y < 0.5\}
$$ 
I use the following scheme. 

To choose a suitable radius to have the particles fill up the available space with a packing fraction of $\rho \approx 1/2$ I solve for $r$ in 
$$
	N \pi r^2 = A(r) \rho = \frac{1}{2} A(r),  
$$
where, in order to avoid having particles outside the box, the area depends on $r$. If we clear a border of $r$ on the boundary of the region facing the walls, we get $r$ from solving
\begin{equation}\label{eq:r}
	N \pi r^2 = \frac{1}{2} \left(1- \frac{1}{2}r\right) \left(1-2r\right) \quad \Rightarrow \quad r = \frac{-1 + \sqrt{N\pi}}{2(N\pi - 1)}.
\end{equation}

To avoid having overlapping particles, I do the following

\begin{algorithm}[H]
	Choose number of particles $N$\;
	Find $r$ corresponding to $N$ from \eqref{eq:r}\;
	$\mathbf{x} \gets [(0,0),\dots,(0,0)]$\;
	Sample $x_i \sim \mathcal{U}_{[r,1-r]}$ and $y_i \sim \mathcal{U}_{[r,0.5]}$\;
	$\mathbf{x}_1 \gets (x_i,y_i)$\;
	\For{$i = 2\dots N$ }{
		Sample $x_i \sim \mathcal{U}_{[r,1-r]}$ and $y_i \sim \mathcal{U}_{[r,0.5]}$\;
		$\mathbf{x}_i \gets (x_i,y_i)$\;
	\While{Particle $i$ does not overlap with particle $1,\dots,i-1$}{
		Sample $x_i \sim \mathcal{U}_{[r,1-r]}$ and $y_i \sim \mathcal{U}_{[r,0.5]}$\;
		$\mathbf{x}_i \gets (x_i,y_i)$\;
	}}
	\caption{Non-overlapping random placement of discs in rectangular region.}
\end{algorithm}  

\subsubsection{Illustration of crater formation}

The plots in figures \ref{fig:crater_1} and \ref{fig:crater_2} shows the crater formed when the mass of the projectile is $5$ and $25$ times the mass of the particles in the bed, respectively. The darker the colour of the particles, the more collisions they have been involved in. Define the \textit{size of the crater}, $\mathcal{S}$, as the number of affected particles by the impact. That is, the number of particles moved during the impact. In the illustrations in figures \ref{fig:crater_1} and \ref{fig:crater_2} the size is the number of non-yellow particles. 

\begin{figure}
	\centering
\begin{minipage}{0.48\columnwidth}
		\includegraphics[width=\linewidth]{../fig/crater_1}
		\captionof{figure}{Example of crater formation using a projectile mass $5$ times the mass of the remaining particles.}
		\label{fig:crater_1}
\end{minipage}
\hfill
\begin{minipage}{0.48\columnwidth}
		\includegraphics[width=\linewidth]{../fig/crater_2}
		\captionof{figure}{Example of crater formation using a projectile mass $25$ times the mass of the remaining particles.}
		\label{fig:crater_2}
\end{minipage}
\end{figure}
\subsubsection{Dependence on projectile mass}

By simulating crater formation with projectile masses running from $1$ to $25$ times the mass of the remaining particles, the size of the crater is as shown in figure \ref{fig:mass_size}. The size of the crater $\mathcal{S}$ is simply the number of particles involved in the crater formation. The plot clearly shows that a larger projectile mass gives rise to a larger crater, as one intuitively would expect. 

\begin{figure}
	\centering
	\includegraphics[width=\columnwidth]{../fig/mass_size}
	\caption{Size of crater as a function of projectile mass. The values are calculated from the mean of simulating projectile impact on $8$ ensembles of a bed of $2000$ particles.}
	\label{fig:mass_size}
\end{figure}

Interestingly, the size seems to scale approximately linearly with the projectile mass. A reasonable suspicion to make is that the size of the crater depends on the energy transferred to it. If this is the case, it would explain the linear dependence on the mass. Following this assumption, one should expect the size to scale quadratically with the initial velocity. However, when trying to demonstrate this I found no such relationship. 
%\subsubsection{Scanning over the projectile velocity}
%
%As alluded to in the previous section, $\mathcal{S} \sim v^2$, as shown in \ref{fig:vel_size}.
%
%\begin{figure}
%	\centering
%	\includegraphics[width=\columnwidth]{../fig/vel_size}
%	\caption{Size of crater as a function of projectile velocity. The values are calculated from the mean of simulating projectile impact on $8$ ensembles of a bed of $2000$ particles.}
%	\label{fig:vel_size}
%\end{figure}


